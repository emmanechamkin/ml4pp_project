
% Default to the notebook output style

    


% Inherit from the specified cell style.




    
\documentclass[11pt]{article}

    
    
    \usepackage[T1]{fontenc}
    % Nicer default font (+ math font) than Computer Modern for most use cases
    \usepackage{mathpazo}

    % Basic figure setup, for now with no caption control since it's done
    % automatically by Pandoc (which extracts ![](path) syntax from Markdown).
    \usepackage{graphicx}
    % We will generate all images so they have a width \maxwidth. This means
    % that they will get their normal width if they fit onto the page, but
    % are scaled down if they would overflow the margins.
    \makeatletter
    \def\maxwidth{\ifdim\Gin@nat@width>\linewidth\linewidth
    \else\Gin@nat@width\fi}
    \makeatother
    \let\Oldincludegraphics\includegraphics
    % Set max figure width to be 80% of text width, for now hardcoded.
    \renewcommand{\includegraphics}[1]{\Oldincludegraphics[width=.8\maxwidth]{#1}}
    % Ensure that by default, figures have no caption (until we provide a
    % proper Figure object with a Caption API and a way to capture that
    % in the conversion process - todo).
    \usepackage{caption}
    \DeclareCaptionLabelFormat{nolabel}{}
    \captionsetup{labelformat=nolabel}

    \usepackage{adjustbox} % Used to constrain images to a maximum size 
    \usepackage{xcolor} % Allow colors to be defined
    \usepackage{enumerate} % Needed for markdown enumerations to work
    \usepackage{geometry} % Used to adjust the document margins
    \usepackage{amsmath} % Equations
    \usepackage{amssymb} % Equations
    \usepackage{textcomp} % defines textquotesingle
    % Hack from http://tex.stackexchange.com/a/47451/13684:
    \AtBeginDocument{%
        \def\PYZsq{\textquotesingle}% Upright quotes in Pygmentized code
    }
    \usepackage{upquote} % Upright quotes for verbatim code
    \usepackage{eurosym} % defines \euro
    \usepackage[mathletters]{ucs} % Extended unicode (utf-8) support
    \usepackage[utf8x]{inputenc} % Allow utf-8 characters in the tex document
    \usepackage{fancyvrb} % verbatim replacement that allows latex
    \usepackage{grffile} % extends the file name processing of package graphics 
                         % to support a larger range 
    % The hyperref package gives us a pdf with properly built
    % internal navigation ('pdf bookmarks' for the table of contents,
    % internal cross-reference links, web links for URLs, etc.)
    \usepackage{hyperref}
    \usepackage{longtable} % longtable support required by pandoc >1.10
    \usepackage{booktabs}  % table support for pandoc > 1.12.2
    \usepackage[inline]{enumitem} % IRkernel/repr support (it uses the enumerate* environment)
    \usepackage[normalem]{ulem} % ulem is needed to support strikethroughs (\sout)
                                % normalem makes italics be italics, not underlines
    \usepackage{mathrsfs}
    

    
    
    % Colors for the hyperref package
    \definecolor{urlcolor}{rgb}{0,.145,.698}
    \definecolor{linkcolor}{rgb}{.71,0.21,0.01}
    \definecolor{citecolor}{rgb}{.12,.54,.11}

    % ANSI colors
    \definecolor{ansi-black}{HTML}{3E424D}
    \definecolor{ansi-black-intense}{HTML}{282C36}
    \definecolor{ansi-red}{HTML}{E75C58}
    \definecolor{ansi-red-intense}{HTML}{B22B31}
    \definecolor{ansi-green}{HTML}{00A250}
    \definecolor{ansi-green-intense}{HTML}{007427}
    \definecolor{ansi-yellow}{HTML}{DDB62B}
    \definecolor{ansi-yellow-intense}{HTML}{B27D12}
    \definecolor{ansi-blue}{HTML}{208FFB}
    \definecolor{ansi-blue-intense}{HTML}{0065CA}
    \definecolor{ansi-magenta}{HTML}{D160C4}
    \definecolor{ansi-magenta-intense}{HTML}{A03196}
    \definecolor{ansi-cyan}{HTML}{60C6C8}
    \definecolor{ansi-cyan-intense}{HTML}{258F8F}
    \definecolor{ansi-white}{HTML}{C5C1B4}
    \definecolor{ansi-white-intense}{HTML}{A1A6B2}
    \definecolor{ansi-default-inverse-fg}{HTML}{FFFFFF}
    \definecolor{ansi-default-inverse-bg}{HTML}{000000}

    % commands and environments needed by pandoc snippets
    % extracted from the output of `pandoc -s`
    \providecommand{\tightlist}{%
      \setlength{\itemsep}{0pt}\setlength{\parskip}{0pt}}
    \DefineVerbatimEnvironment{Highlighting}{Verbatim}{commandchars=\\\{\}}
    % Add ',fontsize=\small' for more characters per line
    \newenvironment{Shaded}{}{}
    \newcommand{\KeywordTok}[1]{\textcolor[rgb]{0.00,0.44,0.13}{\textbf{{#1}}}}
    \newcommand{\DataTypeTok}[1]{\textcolor[rgb]{0.56,0.13,0.00}{{#1}}}
    \newcommand{\DecValTok}[1]{\textcolor[rgb]{0.25,0.63,0.44}{{#1}}}
    \newcommand{\BaseNTok}[1]{\textcolor[rgb]{0.25,0.63,0.44}{{#1}}}
    \newcommand{\FloatTok}[1]{\textcolor[rgb]{0.25,0.63,0.44}{{#1}}}
    \newcommand{\CharTok}[1]{\textcolor[rgb]{0.25,0.44,0.63}{{#1}}}
    \newcommand{\StringTok}[1]{\textcolor[rgb]{0.25,0.44,0.63}{{#1}}}
    \newcommand{\CommentTok}[1]{\textcolor[rgb]{0.38,0.63,0.69}{\textit{{#1}}}}
    \newcommand{\OtherTok}[1]{\textcolor[rgb]{0.00,0.44,0.13}{{#1}}}
    \newcommand{\AlertTok}[1]{\textcolor[rgb]{1.00,0.00,0.00}{\textbf{{#1}}}}
    \newcommand{\FunctionTok}[1]{\textcolor[rgb]{0.02,0.16,0.49}{{#1}}}
    \newcommand{\RegionMarkerTok}[1]{{#1}}
    \newcommand{\ErrorTok}[1]{\textcolor[rgb]{1.00,0.00,0.00}{\textbf{{#1}}}}
    \newcommand{\NormalTok}[1]{{#1}}
    
    % Additional commands for more recent versions of Pandoc
    \newcommand{\ConstantTok}[1]{\textcolor[rgb]{0.53,0.00,0.00}{{#1}}}
    \newcommand{\SpecialCharTok}[1]{\textcolor[rgb]{0.25,0.44,0.63}{{#1}}}
    \newcommand{\VerbatimStringTok}[1]{\textcolor[rgb]{0.25,0.44,0.63}{{#1}}}
    \newcommand{\SpecialStringTok}[1]{\textcolor[rgb]{0.73,0.40,0.53}{{#1}}}
    \newcommand{\ImportTok}[1]{{#1}}
    \newcommand{\DocumentationTok}[1]{\textcolor[rgb]{0.73,0.13,0.13}{\textit{{#1}}}}
    \newcommand{\AnnotationTok}[1]{\textcolor[rgb]{0.38,0.63,0.69}{\textbf{\textit{{#1}}}}}
    \newcommand{\CommentVarTok}[1]{\textcolor[rgb]{0.38,0.63,0.69}{\textbf{\textit{{#1}}}}}
    \newcommand{\VariableTok}[1]{\textcolor[rgb]{0.10,0.09,0.49}{{#1}}}
    \newcommand{\ControlFlowTok}[1]{\textcolor[rgb]{0.00,0.44,0.13}{\textbf{{#1}}}}
    \newcommand{\OperatorTok}[1]{\textcolor[rgb]{0.40,0.40,0.40}{{#1}}}
    \newcommand{\BuiltInTok}[1]{{#1}}
    \newcommand{\ExtensionTok}[1]{{#1}}
    \newcommand{\PreprocessorTok}[1]{\textcolor[rgb]{0.74,0.48,0.00}{{#1}}}
    \newcommand{\AttributeTok}[1]{\textcolor[rgb]{0.49,0.56,0.16}{{#1}}}
    \newcommand{\InformationTok}[1]{\textcolor[rgb]{0.38,0.63,0.69}{\textbf{\textit{{#1}}}}}
    \newcommand{\WarningTok}[1]{\textcolor[rgb]{0.38,0.63,0.69}{\textbf{\textit{{#1}}}}}
    
    
    % Define a nice break command that doesn't care if a line doesn't already
    % exist.
    \def\br{\hspace*{\fill} \\* }
    % Math Jax compatibility definitions
    \def\gt{>}
    \def\lt{<}
    \let\Oldtex\TeX
    \let\Oldlatex\LaTeX
    \renewcommand{\TeX}{\textrm{\Oldtex}}
    \renewcommand{\LaTeX}{\textrm{\Oldlatex}}
    % Document parameters
    % Document title
    \title{Measuring Public Sentiment towards the Police}
    \author{Emma Nechamkin}
    
    
    
    

    % Pygments definitions
    
\makeatletter
\def\PY@reset{\let\PY@it=\relax \let\PY@bf=\relax%
    \let\PY@ul=\relax \let\PY@tc=\relax%
    \let\PY@bc=\relax \let\PY@ff=\relax}
\def\PY@tok#1{\csname PY@tok@#1\endcsname}
\def\PY@toks#1+{\ifx\relax#1\empty\else%
    \PY@tok{#1}\expandafter\PY@toks\fi}
\def\PY@do#1{\PY@bc{\PY@tc{\PY@ul{%
    \PY@it{\PY@bf{\PY@ff{#1}}}}}}}
\def\PY#1#2{\PY@reset\PY@toks#1+\relax+\PY@do{#2}}

\expandafter\def\csname PY@tok@w\endcsname{\def\PY@tc##1{\textcolor[rgb]{0.73,0.73,0.73}{##1}}}
\expandafter\def\csname PY@tok@c\endcsname{\let\PY@it=\textit\def\PY@tc##1{\textcolor[rgb]{0.25,0.50,0.50}{##1}}}
\expandafter\def\csname PY@tok@cp\endcsname{\def\PY@tc##1{\textcolor[rgb]{0.74,0.48,0.00}{##1}}}
\expandafter\def\csname PY@tok@k\endcsname{\let\PY@bf=\textbf\def\PY@tc##1{\textcolor[rgb]{0.00,0.50,0.00}{##1}}}
\expandafter\def\csname PY@tok@kp\endcsname{\def\PY@tc##1{\textcolor[rgb]{0.00,0.50,0.00}{##1}}}
\expandafter\def\csname PY@tok@kt\endcsname{\def\PY@tc##1{\textcolor[rgb]{0.69,0.00,0.25}{##1}}}
\expandafter\def\csname PY@tok@o\endcsname{\def\PY@tc##1{\textcolor[rgb]{0.40,0.40,0.40}{##1}}}
\expandafter\def\csname PY@tok@ow\endcsname{\let\PY@bf=\textbf\def\PY@tc##1{\textcolor[rgb]{0.67,0.13,1.00}{##1}}}
\expandafter\def\csname PY@tok@nb\endcsname{\def\PY@tc##1{\textcolor[rgb]{0.00,0.50,0.00}{##1}}}
\expandafter\def\csname PY@tok@nf\endcsname{\def\PY@tc##1{\textcolor[rgb]{0.00,0.00,1.00}{##1}}}
\expandafter\def\csname PY@tok@nc\endcsname{\let\PY@bf=\textbf\def\PY@tc##1{\textcolor[rgb]{0.00,0.00,1.00}{##1}}}
\expandafter\def\csname PY@tok@nn\endcsname{\let\PY@bf=\textbf\def\PY@tc##1{\textcolor[rgb]{0.00,0.00,1.00}{##1}}}
\expandafter\def\csname PY@tok@ne\endcsname{\let\PY@bf=\textbf\def\PY@tc##1{\textcolor[rgb]{0.82,0.25,0.23}{##1}}}
\expandafter\def\csname PY@tok@nv\endcsname{\def\PY@tc##1{\textcolor[rgb]{0.10,0.09,0.49}{##1}}}
\expandafter\def\csname PY@tok@no\endcsname{\def\PY@tc##1{\textcolor[rgb]{0.53,0.00,0.00}{##1}}}
\expandafter\def\csname PY@tok@nl\endcsname{\def\PY@tc##1{\textcolor[rgb]{0.63,0.63,0.00}{##1}}}
\expandafter\def\csname PY@tok@ni\endcsname{\let\PY@bf=\textbf\def\PY@tc##1{\textcolor[rgb]{0.60,0.60,0.60}{##1}}}
\expandafter\def\csname PY@tok@na\endcsname{\def\PY@tc##1{\textcolor[rgb]{0.49,0.56,0.16}{##1}}}
\expandafter\def\csname PY@tok@nt\endcsname{\let\PY@bf=\textbf\def\PY@tc##1{\textcolor[rgb]{0.00,0.50,0.00}{##1}}}
\expandafter\def\csname PY@tok@nd\endcsname{\def\PY@tc##1{\textcolor[rgb]{0.67,0.13,1.00}{##1}}}
\expandafter\def\csname PY@tok@s\endcsname{\def\PY@tc##1{\textcolor[rgb]{0.73,0.13,0.13}{##1}}}
\expandafter\def\csname PY@tok@sd\endcsname{\let\PY@it=\textit\def\PY@tc##1{\textcolor[rgb]{0.73,0.13,0.13}{##1}}}
\expandafter\def\csname PY@tok@si\endcsname{\let\PY@bf=\textbf\def\PY@tc##1{\textcolor[rgb]{0.73,0.40,0.53}{##1}}}
\expandafter\def\csname PY@tok@se\endcsname{\let\PY@bf=\textbf\def\PY@tc##1{\textcolor[rgb]{0.73,0.40,0.13}{##1}}}
\expandafter\def\csname PY@tok@sr\endcsname{\def\PY@tc##1{\textcolor[rgb]{0.73,0.40,0.53}{##1}}}
\expandafter\def\csname PY@tok@ss\endcsname{\def\PY@tc##1{\textcolor[rgb]{0.10,0.09,0.49}{##1}}}
\expandafter\def\csname PY@tok@sx\endcsname{\def\PY@tc##1{\textcolor[rgb]{0.00,0.50,0.00}{##1}}}
\expandafter\def\csname PY@tok@m\endcsname{\def\PY@tc##1{\textcolor[rgb]{0.40,0.40,0.40}{##1}}}
\expandafter\def\csname PY@tok@gh\endcsname{\let\PY@bf=\textbf\def\PY@tc##1{\textcolor[rgb]{0.00,0.00,0.50}{##1}}}
\expandafter\def\csname PY@tok@gu\endcsname{\let\PY@bf=\textbf\def\PY@tc##1{\textcolor[rgb]{0.50,0.00,0.50}{##1}}}
\expandafter\def\csname PY@tok@gd\endcsname{\def\PY@tc##1{\textcolor[rgb]{0.63,0.00,0.00}{##1}}}
\expandafter\def\csname PY@tok@gi\endcsname{\def\PY@tc##1{\textcolor[rgb]{0.00,0.63,0.00}{##1}}}
\expandafter\def\csname PY@tok@gr\endcsname{\def\PY@tc##1{\textcolor[rgb]{1.00,0.00,0.00}{##1}}}
\expandafter\def\csname PY@tok@ge\endcsname{\let\PY@it=\textit}
\expandafter\def\csname PY@tok@gs\endcsname{\let\PY@bf=\textbf}
\expandafter\def\csname PY@tok@gp\endcsname{\let\PY@bf=\textbf\def\PY@tc##1{\textcolor[rgb]{0.00,0.00,0.50}{##1}}}
\expandafter\def\csname PY@tok@go\endcsname{\def\PY@tc##1{\textcolor[rgb]{0.53,0.53,0.53}{##1}}}
\expandafter\def\csname PY@tok@gt\endcsname{\def\PY@tc##1{\textcolor[rgb]{0.00,0.27,0.87}{##1}}}
\expandafter\def\csname PY@tok@err\endcsname{\def\PY@bc##1{\setlength{\fboxsep}{0pt}\fcolorbox[rgb]{1.00,0.00,0.00}{1,1,1}{\strut ##1}}}
\expandafter\def\csname PY@tok@kc\endcsname{\let\PY@bf=\textbf\def\PY@tc##1{\textcolor[rgb]{0.00,0.50,0.00}{##1}}}
\expandafter\def\csname PY@tok@kd\endcsname{\let\PY@bf=\textbf\def\PY@tc##1{\textcolor[rgb]{0.00,0.50,0.00}{##1}}}
\expandafter\def\csname PY@tok@kn\endcsname{\let\PY@bf=\textbf\def\PY@tc##1{\textcolor[rgb]{0.00,0.50,0.00}{##1}}}
\expandafter\def\csname PY@tok@kr\endcsname{\let\PY@bf=\textbf\def\PY@tc##1{\textcolor[rgb]{0.00,0.50,0.00}{##1}}}
\expandafter\def\csname PY@tok@bp\endcsname{\def\PY@tc##1{\textcolor[rgb]{0.00,0.50,0.00}{##1}}}
\expandafter\def\csname PY@tok@fm\endcsname{\def\PY@tc##1{\textcolor[rgb]{0.00,0.00,1.00}{##1}}}
\expandafter\def\csname PY@tok@vc\endcsname{\def\PY@tc##1{\textcolor[rgb]{0.10,0.09,0.49}{##1}}}
\expandafter\def\csname PY@tok@vg\endcsname{\def\PY@tc##1{\textcolor[rgb]{0.10,0.09,0.49}{##1}}}
\expandafter\def\csname PY@tok@vi\endcsname{\def\PY@tc##1{\textcolor[rgb]{0.10,0.09,0.49}{##1}}}
\expandafter\def\csname PY@tok@vm\endcsname{\def\PY@tc##1{\textcolor[rgb]{0.10,0.09,0.49}{##1}}}
\expandafter\def\csname PY@tok@sa\endcsname{\def\PY@tc##1{\textcolor[rgb]{0.73,0.13,0.13}{##1}}}
\expandafter\def\csname PY@tok@sb\endcsname{\def\PY@tc##1{\textcolor[rgb]{0.73,0.13,0.13}{##1}}}
\expandafter\def\csname PY@tok@sc\endcsname{\def\PY@tc##1{\textcolor[rgb]{0.73,0.13,0.13}{##1}}}
\expandafter\def\csname PY@tok@dl\endcsname{\def\PY@tc##1{\textcolor[rgb]{0.73,0.13,0.13}{##1}}}
\expandafter\def\csname PY@tok@s2\endcsname{\def\PY@tc##1{\textcolor[rgb]{0.73,0.13,0.13}{##1}}}
\expandafter\def\csname PY@tok@sh\endcsname{\def\PY@tc##1{\textcolor[rgb]{0.73,0.13,0.13}{##1}}}
\expandafter\def\csname PY@tok@s1\endcsname{\def\PY@tc##1{\textcolor[rgb]{0.73,0.13,0.13}{##1}}}
\expandafter\def\csname PY@tok@mb\endcsname{\def\PY@tc##1{\textcolor[rgb]{0.40,0.40,0.40}{##1}}}
\expandafter\def\csname PY@tok@mf\endcsname{\def\PY@tc##1{\textcolor[rgb]{0.40,0.40,0.40}{##1}}}
\expandafter\def\csname PY@tok@mh\endcsname{\def\PY@tc##1{\textcolor[rgb]{0.40,0.40,0.40}{##1}}}
\expandafter\def\csname PY@tok@mi\endcsname{\def\PY@tc##1{\textcolor[rgb]{0.40,0.40,0.40}{##1}}}
\expandafter\def\csname PY@tok@il\endcsname{\def\PY@tc##1{\textcolor[rgb]{0.40,0.40,0.40}{##1}}}
\expandafter\def\csname PY@tok@mo\endcsname{\def\PY@tc##1{\textcolor[rgb]{0.40,0.40,0.40}{##1}}}
\expandafter\def\csname PY@tok@ch\endcsname{\let\PY@it=\textit\def\PY@tc##1{\textcolor[rgb]{0.25,0.50,0.50}{##1}}}
\expandafter\def\csname PY@tok@cm\endcsname{\let\PY@it=\textit\def\PY@tc##1{\textcolor[rgb]{0.25,0.50,0.50}{##1}}}
\expandafter\def\csname PY@tok@cpf\endcsname{\let\PY@it=\textit\def\PY@tc##1{\textcolor[rgb]{0.25,0.50,0.50}{##1}}}
\expandafter\def\csname PY@tok@c1\endcsname{\let\PY@it=\textit\def\PY@tc##1{\textcolor[rgb]{0.25,0.50,0.50}{##1}}}
\expandafter\def\csname PY@tok@cs\endcsname{\let\PY@it=\textit\def\PY@tc##1{\textcolor[rgb]{0.25,0.50,0.50}{##1}}}

\def\PYZbs{\char`\\}
\def\PYZus{\char`\_}
\def\PYZob{\char`\{}
\def\PYZcb{\char`\}}
\def\PYZca{\char`\^}
\def\PYZam{\char`\&}
\def\PYZlt{\char`\<}
\def\PYZgt{\char`\>}
\def\PYZsh{\char`\#}
\def\PYZpc{\char`\%}
\def\PYZdl{\char`\$}
\def\PYZhy{\char`\-}
\def\PYZsq{\char`\'}
\def\PYZdq{\char`\"}
\def\PYZti{\char`\~}
% for compatibility with earlier versions
\def\PYZat{@}
\def\PYZlb{[}
\def\PYZrb{]}
\makeatother


    % Exact colors from NB
    \definecolor{incolor}{rgb}{0.0, 0.0, 0.5}
    \definecolor{outcolor}{rgb}{0.545, 0.0, 0.0}



    
    % Prevent overflowing lines due to hard-to-break entities
    \sloppy 
    % Setup hyperref package
    \hypersetup{
      breaklinks=true,  % so long urls are correctly broken across lines
      colorlinks=true,
      urlcolor=urlcolor,
      linkcolor=linkcolor,
      citecolor=citecolor,
      }
    % Slightly bigger margins than the latex defaults
    
    \geometry{verbose,tmargin=1in,bmargin=1in,lmargin=1in,rmargin=1in}
    
    

    \begin{document}
    
    
    \maketitle
    
    

    
    \hypertarget{introduction}{%
\section{Introduction}\label{introduction}}

Public perception of the police is incredibly important to police
effectiveness and legitimacy but extremely difficult to measure. Public
perception offers insight into how well a police department is
functioning and may suggest adherence to tenets of procedural justice.
Yet, compared to traditional performance metrics, metrics to evaluate
public opinion are poorly defined and documented.

\hypertarget{clearance-rates}{%
\subsection{Clearance rates}\label{clearance-rates}}

Homicide clearance rates, or what share of murders a police department
``solves'', are a key performance metric for police departments. Chicago
has one of the lowest homicide clearance rates in the country, and only
about 1 in 6 murders lead to arrest. Moreover, Chicago's clearance rate
has steadily declined over the past ten years, from about 40\% in 2000
down to under 20\% in 2017. As a comparison, several police departments
have markedly higher clearance rates. Over the past decade, Los Angeles
has solved 51\% of murders and New York has solved 61\% of murders.

There are several potential reasons for the low clearance rate in
Chicago, some of which suggest that non-traditional metrics of policing
like procedural justice or public opinion may be related to traditional
metrics. Police officers tend to cite the historically fraught
relationship between the people and police, believing that someone who
already views the police negatively because police seem inept may be
less likely to cooperate with an investigation; more bluntly, many
police officers lament a ``no snitch'' policy among victimized
communities in Chicago. Evidence is conflicted: the National Crime
Victimization Survey reports that these communities are no less likely
to report crimes to the police, but a Cato Institute survey shows a race
and education gap for crime reporting. There are also other viable
explanations for Chicago's abysmal clearance rate, most notably that
Chicago's police force has limited manpower per murder. Chicago has more
murders than New York and Los Angeles combined, yet the police
department (12,000 officers) is dwarved by New York's (36,000) and Los
Angeles' (10,000).

\hypertarget{public-opinion}{%
\subsection{Public opinion}\label{public-opinion}}

Procedural justice, or how police
officers enforce laws, is necessary for effective policing. A civilian who
considers the law enforcement process fair and just is likely to
consider any related consequences fair and just, too. Conversely, when
civilians perceive lack of procedural justice, they are more likely to
file complaints and view their police force as delegitimate. For
example, one study of New York Police Department Stop, Question, and
Frisk stops showed that civilians who believed their stop to be fair
were less likely to file a complaint than those who believed their stop
was unjust. Finally, a lack of procedural justice in just a few
encouters can severely curtail public opinion of the police. Negative
interactions with the police shape citizen perception up to fourteen
times more strongly than positive ones.

Public perception of the police offers an additional metric to assess
police performance. While hard metrics like clearance rates are easy to
measure, assessing how the public feels towards the police is far more
complex. Indeed, most work that tries to assess public sentiment uses
survey-based or experimental research. Indeed, most past research has evaluated 
procedural justice through
the lens of public opinion survey data. Such research is necessarily
removed from the real world.

\hypertarget{research-goals}{%
\subsection{Research goals}\label{research-goals}}

Although public perception of the police is complex, I'm interested in
assessing whether public perception of the police in related tweets is
correlated with police effectiveness as measured by clearance rates. 
This follows recent work that has considered sentiments
of tweets to assess public opinion of the police.
  
More specifically, I'm interested in assessing, in order of importance:

\begin{enumerate}
\def\labelenumi{\arabic{enumi}.}
\tightlist
\item
  The extent to which public sentiment reflects traditional metrics of
  police effectiveness,
\item
  The differences by topic between effective and less effective police
  departments, and
\item
  How tweets about police departments reflect tenets of procedural
  justice.
\end{enumerate}

As a caveat, public
perception of the police is complicated and interacts with policing in
myriad ways.

\hypertarget{past-work}{%
\section{Past Work}\label{past-work}}

\hypertarget{using-twitter-data-to-measure-public-sentiment-towards-the-police}{%
\subsection{Using twitter data to measure public sentiment towards the
police}\label{using-twitter-data-to-measure-public-sentiment-towards-the-police}}

Although there has been limited work using data science techniques to
study criminal justice, the Urban Institute used sentiment analysis for
police-related tweets to measure how perception of the police changed
due to the murder of Freddie Gray, using the following methods:

\begin{itemize}
\tightlist
\item
  Obtaining the data: researchers used a set of relevant tweets from
  2014 and 2015 acquired through twitter.
\item
  Processing the data: researchers removed mentions, hashtags, links,
  punctuation, and stop words from all tweets. They also used CoreNLP to
  tag tweets (e.g., to identify whether ``cop'' was a noun or a verb in
  each tweet).
\item
  Learning models: researchers classified over 4,000 tweets manually to
  identify whether the tweet was positive, negative, neutral, or not
  applicable to their research for use in training and validation sets.
  They then used several types of models to predict the sentiment of new
  tweets and selected a gradient-boosted regression classifier as their
  model based on its accuracy (63\%).
\item
  Conclusions: researchers then used their newly labeled set of all
  tweets to assess the shift in public sentiment over time.
\end{itemize}

\hypertarget{using-twitter-data-to-connect-public-opinion-with-tweet-sentiment}{%
\subsection{Using twitter data to connect public opinion with tweet
sentiment}\label{using-twitter-data-to-connect-public-opinion-with-tweet-sentiment}}

As a more general example, researchers at Carnegie Mellon University
determined that public opinion surveys correlate to twitter sentiment on
several key issues. They used twitter data specifically with two
endgoals: to identify relevant tweets and to estimate sentiment
(positive and negative) about a given topic.

In their work, researchers obtained tweets from 2008 and 2009 using the
twitter API. They then used key words (like ``obama'' to measure
presidential approval) to ensure that their tweets were relevant. Tweets
were classified as positive, negative, or both depending on whether
there was a positive, a negative, or both types of words in it. Finally,
to get an accurate measure of sentiment, they computed a moving average
aggregate of sentiment ratios, where sentiment ratio was defined as the
ratio between the number of positive and negative relevant tweets. The
moving averages allowed them to smooth otherwise volatile data. They
then investigated correlations between the sentiment they uncovered and
traditional public opinion surveys.

\hypertarget{logistics}{%
\section{Logistics}\label{logistics}}

\hypertarget{steps}{%
\subsection{Steps}\label{steps}}

For this project, I will specifically focus on Chicago, Los Angeles, and
New York in several (broad stroke) steps.

\begin{itemize}
\tightlist
\item
  Obtain relevant data using the twitter API and/or python's tweepy package. Relevant tweets will
  contain hashtags related to Chicago police, New York police, and Los
  Angeles police
\item
  Pre-process data by removing stop words and using baseline CoreNLP
  functionality and potentially python's NLTK package. 
\item
  Create a model to identify the sentiment of tweets (TBD based on results)
\item
  Use topic modeling to evaluate how procedural justice is reflected in
  tweets, and what else is reflected in relevant tweets
\end{itemize}

\hypertarget{packages}{%
\subsection{Packages}\label{packages}}

I plan to use an AWS server to complete the CoreNLP work, and a variety
of ML packages in python to build and evaluate models (include
scikit-learn, and, depending on the models used, other more advanced
packages). I also plan to use python's NLTK for topic modeling and tweepy/twitter's API.

\hypertarget{milestones}{%
\subsection{Milestones}\label{milestones}}

At minimum, I hope to have completed the first two steps listed
(obtaining relevant tweets and preprocessing tweets using CoreNLP and
manual tagging) by the mid-quarter presentation.

\hypertarget{project-evaluation}{%
\subsection{Project evaluation}\label{project-evaluation}}

By the end of the project, I hope to evaluate the sentiment using a
testing / training / validation framework in which I assess how well 
my model has classified tweets. 

Topic modeling is currently an aspirational part of this project and I expect
to have a more concrete plan as we gain depth in this class. 

\hypertarget{logistics-1}{%
\subsection{Logistics}\label{logistics-1}}

I am working alone and so do not need to coordinate work among
teammates. My rough timeline is shown below. This timeline is
aspirational and reflects the ``best case'' of work.

\begin{itemize}
\tightlist
\item
  April 19: Twitter data collected
\item
  April 23: Basic preprocessing completed (CoreNLP tagging, similar to
  Urban Institute)
\item
  April 26: Manual coding of tweet sentiment completed
\item
  April 31: First run of selections of models completed
\item
  May 3: Models evaluated
\item
  May 10: Model refinement and adjustment
\item
  May 15: Topic modeling outlined
\item
  May 20: Topic modeling completed
\item
  May 25: Analysis completed
\item
  May 30: Report and presentation draft completed
\end{itemize}

    \hypertarget{selected-sources}{%
\section{Selected sources (informal)}\label{selected-sources}}

\hypertarget{papers}{%
\subsection{Papers}\label{papers}}
\begin{itemize}
\tightlist

\item
Ekins, Emily. (2016). Policing in America: Understanding Public
Attitudes Toward the Police. Results from a National Survey. SSRN
Electronic Journal. 10.2139/ssrn.2919449.

\item
Fowler, AF Rengifo and K. 2016. ``Stop, Question, and Complain: Citizen
Grievances Against the NYPD and the Opacity of Police Stops Across New
York City Precincts, 2007-2013.'' Journal of Urban Health (93 Suppl 1):
32-41.

\item
O'Connor, Brendan \& Balasubramanyan, Ramnath \& R. Routledge, Bryan \&
A. Smith, Noah. (2010). From Tweets to Polls: Linking Text Sentiment to
Public Opinion Time Series. International AAAI Conference on Weblogs and
Social Media. 11.

\item
Oglesby-Neal, Ashlin, Tiry, Emily, and Kim, KiDeuk. ``Public Perceptions of Police
on Social Media." 19 February 2019. Urban Institute Research Brief. 

\item
Skogan, Wesley G. 2006. ``Asymmetry in the Impact of Encounters With
Police.'' Policing and Society.

\item
Tyler, Tom R. 2004. ``Enhancing Police Legitimacy.'' The Annals of the
American Academy of Political and Social Science 593: 84-99.
\end{itemize}

\hypertarget{news-and-articles-quick-links}{%
\subsection{News and articles (quick
links)}\label{news-and-articles-quick-links}}

\begin{itemize}
\tightlist
\item
  https://www.washingtonpost.com/graphics/2018/investigations/unsolved-homicide-database/?utm\_term=.8da8a801878a\&city=indianapolis{]}
\item
  https://chicago.suntimes.com/news/murder-clearance-rate-in-chicago-hit-new-low-in-2017/
\item
  https://www.theatlantic.com/ideas/archive/2018/05/quis-custodiet-ipsos-custodes/560324/
\item
  https://datasmart.ash.harvard.edu/news/article/map-monday-unsolved-homicides
\end{itemize}
    
    
    
    \end{document}
